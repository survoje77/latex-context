\enableregime[utf-8]
\useencoding[utf-8]
\mainlanguage[fr]
%-----MODULES
\usemodule[tikz]
%----booklet A6------------
% en supprimant le % devant [A4] et devant %%\setuparranging on obtient un livret 
 \setuppapersize[A6]%[A4]
%%\setuparranging[2*4,doublesided]
\setuplayout[location=middle,
%backspace=10mm,
backspace=10mm,
% leftmargin=3mm,
 leftmargin=3mm,
%  leftmargindistance=5mm,
%  rightmargindistance=0mm,
 rightmargin=0mm,
%  rightedge=0mm,
   width=90mm,
   topspace=5mm,
   header=5mm,
    headerdistance=5mm,
    footer=0mm,
   height=fit,
   bottom=0mm
]
\setuppagenumbering[alternative=doublesided]
%-----------------------------------
\setupitemize[2]
\setupframed[backgroundcolor=red]
\setupframedtexts[width=8cm]
%\setupbodyfont[handwritten]

\setupcolors[state=start]

%--------COULEURS-----------
\definecolor[monbleu][r=.0,g=.0,b=.7]
\definecolor[monrouge][r=.5,g=.0,b=.0]
\definecolor[monvert][r=.0,g=.3,b=.0]
\definecolor[darkgreen][g=.5]%exemple tikz

\define[1]\grosbleu
{{\tfc \color[monbleu]{#1}}}
\define[1]\grosnoir
{{\tfc \color[black]{#1}}}
\define[1]\grosrouge
{{\tfc \color[monrouge]{#1}}}
\define[1]\grosvert
{{\tfc \color[monvert]{#1}}}

\define[1]\tibleu
{{\bf \color[monbleu]{#1}}}
\define[1]\tinoir
{{\bf  \color[black]{#1}}}
\define[1]\tirouge
{{\bf \color[monrouge]{#1}}}
\define[1]\tivert
{{\bf \color[monvert]{#1}}}

\define[1]\bleu
{{\tfb \color[monbleu]{#1}}}
\define[1]\noir
{{\tfb \color[black]{#1}}}
\define[1]\rouge
{{\tfb \color[monrouge]{#1}}}
\define[1]\vert
{{\tfb \color[monvert]{#1}}}

\definecolor[bleuLille][r=.1,g=0.47,b=.72]

\definefloat[intermezzo]

\definebodyfont[10pt,11pt,12pt][rm][tfe=Regular at 48pt]
%-----mes macros
\define[2]\moncadre
{ \startframedtext[middle][width=9cm,corner=round,framecolor=#1]
#2
\stopframedtext
}

\define\titre
{ \definedfont[SansBold at 20pt] \black{Être }\definedfont[SansBold at 28pt]\bleuLille{acteur }\definedfont[SansBold at 20pt]\black{de sa vie} \crlf
 \definedfont[Sans at 16pt] \monrouge{Krishnamurti}
  \blank[3*big]
}

\define[2]\police
{\definedfont[#1]\black{#2}\normal
}
%-----------police sans
\define[2]\sans
{{\definedfont[Sans at #1]{#2}}}

\define[2]\sansB
{{\definedfont[SansBold at #1]{#2}}}

\define[2]\sansI
{{\definedfont[SansItalic at #1]{#2}}}

\define[2]\sansBI
{{\definedfont[SansBoldItalic at #1]{#2}}}
%------------police serif
\define[2]\serif
{{\definedfont[Serif at #1]{#2}}}

\define[2]\serifB
{{\definedfont[SerifBold at #1]{#2}}}

\define[2]\serifI
{{\definedfont[SerifItalic at #1]{#2}}}

\define[2]\serifBI
{{\definedfont[SerifBoldItalic at #1]{#2}}}
%----police mono
\define[2]\mono
{{\definedfont[Mono at #1]{#2}}}

\define[2]\monoB
{{\definedfont[MonoBold at #1]{#2}}}

\define[2]\monoI
{{\definedfont[MonoItalic at #1]{#2}}}

\define[2]\monoBI
{{\definedfont[MonoBoldItalic at #1]{#2}}}

%----LAYERS------
 \definelayer
  [cadre]
  [x=0mm,y=0mm,width=\textwidth,height=\textheight]
 \setlayerframed
   [cadre]
   [hoffset=0cm, voffset=0cm,height=15.5cm,width=1.5cm,frame=off,
   background=color,backgroundcolor=bleuLille]
   {}

 \definelayer[rectangle][x=0mm,y=0mm,width=\paperwidth, height=\paperheight]

\definelayer[texte] [x=0mm, y=0mm, width=\paperwidth, height=\paperheight]

 \definelayer
  [pensee]
  [x=0mm,y=0mm,width=\textwidth,height=\textheight]

%-------bulle pour tikz
  \definelayer
  [bulle]
  [x=0mm,y=0mm,width=\textwidth,height=\textheight]


\setupbackgrounds[page][background=texte]



 \setuptextrules[color=red,rulecolor=blue]

  \setlayerframed[bulle][hoffset=5cm, voffset=9.5cm,frame=off]
 {\rotate[rotation=20]{
 \starttikzpicture
 \shade[ball color=monrouge,opacity=.6] (0,0) circle (5ex);
 \shade[ball color=monrouge,opacity=.8] (0.8,0.5) circle (2.5ex);
 \shade[ball color=monrouge,opacity=.8] (0.7,1.5) circle (1ex);
 \shade[ball color=bleuLille,opacity=.5] (-0.25,1.5) circle (0.5ex);
 \shade[ball color=bleuLille,opacity=.8] (0.5,2) circle (0.5ex);
 \shade[ball color=bleuLille,opacity=.6] (1.8,1.4) circle (1ex);
 \shade[ball color=monrouge,opacity=.8] (1.2,2.3) circle (2ex);
 \shade[ball color=bleuLille,opacity=.5] (1.5,2) circle (1.5ex);
 \shade[ball color=monrouge,opacity=.8] (2.25,2.25) circle (1.5ex);
 \shade[ball color=bleuLille,opacity=.5] (2.5,2.3) circle (1ex);
 \shade[ball color=bleuLille,opacity=.8] (2.5,3) circle (0.75ex);
 \shade[ball color=bleuLille,opacity=.6] (1.75,2.25) circle (1ex);
 \shade[ball color=bleuLille,opacity=.8] (2.5,1.5) circle (0.5ex);
 \stoptikzpicture}}

%-----DOCUMENT-----------------------
  \starttext
  %\showlayout
  %----couverture
 \setupframed[backgroundcolor=blue,framecolor=bleuLille]
\setupcolors[state=start]
\setuppagenumbering[state=stop]
\setupbackgrounds[page][background={cadre,pensee,bulle}]
%\setupbackgrounds[paper][state=repeat,background={bulle}]


 \setlayerframed
   [cadre]
   [hoffset=0cm, voffset=0cm,height=14.9cm,width=1.5cm,frame=off,
   background=color,backgroundcolor=bleuLille]
   {}
     \setlayerframed
  [pensee]
  [hoffset=-1.75cm, voffset=13.5cm,frame=off,width=5cm]
  {\white\rotate[rotation=90]{\definedfont[SansBold at 16pt]  A6} }

   %-----TITRE------
\startcolor[bleuLille]
\startalignment[left]
\titre
\placefigure
	[left]
	[fig:K]
	{none}
	\framed{\externalfigure[krishnamurti.jpg]
	[width=.6\textwidth]}
\stopcolor
 \stopalignment
%\-------page 2
\page
\setuppagenumber[number=2]
\setuppagenumbering[state=start]

\startalignment[center]
\serif{16pt}{La question de l'}\sansB{36pt}{action}
%\stopalignment
\serif{14pt}{ doit absolument être au rang de nos}
%\startalignment[left]
\police{SansBold at 20pt}{préoccupations majeures}
\crlf \crlf\crlf
\stopalignment
%\page
\startalignment[center]
\sansB{12pt}{Confrontés} \sans{12pt}{comme nous le sommes}

\sans{12pt}{à une }\sansB{12pt}{multitude}

\sans{12pt}{de }\sansB{18pt}{problèmes }

 \sans{12pt}{ comme la}  \sans{20pt}{pauvreté},

 \sans{12pt}{la} \sans{20pt}{surpopulation},

 \sans{12pt}{la} \sans{20pt}{mécanisation},

 \sans{12pt}{l'}\sans{20pt}{industrialisation}

 \sans{12pt}{et ce sentiment d'une} \sans{16pt}{détérioration}

\sans{20pt}{intérieure et extérieure}, \crlf

\sans{48pt}{que faire ?}
\stopalignment
%------page 3
\page
\startalignment[left]
\sans{18pt}{Ce que vous devez}\sansB{20pt}{ faire}

\sans{18pt}{ n'a pas d'}\serifB{20pt}{importance}

\sans{18pt}{mais il est essentiel}

 \sans{20pt}{d'avoir}\sansB{20pt}{ conscience}

\sans{18pt}{de ce que vous êtes}

 \sans{18pt}{en train de} \sansB{20pt}{faire}
 \stopalignment
\crlf

Il est très facile de se noyer dans l'\serifB{16pt}{activisme} ou les \serifB{16pt}{réformes sociales}, mais je ne pense pas que cela résolve les nombreux \serifB{16pt}{problèmes} qui nous assaillent. Nous avons besoin d'une\crlf
{\midaligned{ \serifB{20pt}{réponse de fond}}\crlf\crlf\crlf
\sans{12pt}{Notre action résulte de notre} \sansB{16pt}{conditionnement}\crlf\crlf
\serif{16pt}{Pour la plupart d'entre nous},

\serif{16pt}{l'}\serif{20pt}{activité} \serif{16pt}{sous toutes ses formes}

\serif{16pt}{est une} \serif{24pt}{drogue}\crlf\crlf
%----page4
\page
\startalignment[middle]
\serifI{12pt}{Jamais le cerveau }

\serifI{12pt}{ne goûte un moment de}

\serifI{20pt}{repos},

\serifI{12pt}{jamais il ne peut dire}

\serifI{20pt}{\quotation{j'ai fini}}
 \stopalignment
\crlf%\crlf
 \sans{12pt}{L'esprit n'est qu'une suite de réactions}

 \sans{12pt}{et la} \sans{20pt}{révolution}

 \sans{12pt}{établie sur les réactions, }

 \sans{12pt}{sur les idées,}  \sans{12pt}{n'est pas une}  \sans{20pt}{révolution}

  \sans{12pt}{mais tout au plus}  \sans{16pt}{une continuité modifiée}
\crlf\crlf

\serif{16pt}{Peut-il y avoir une} \serifB{16pt}{action} \serif{16pt}{sans connexion avec le} \serifB{16pt}{passé}\serif{16pt}{, qui ne soit pas plombée par le fardeau de l'}\serifB{16pt}{expérience}\serif{16pt}{, par le }\serifB{16pt}{savoir} \serif{16pt}{ d'hier ?}
\crlf\crlf\crlf\crlf
\startalignment[middle]
\sans{16pt}{Nos actions ne sont que des}
\sans{26pt}{réactions}
 \stopalignment
 %----page 5
 \page
\mono{20pt}{L'esprit peut-il }\crlf
\monoB{26pt}{se libérer }\crlf
 \mono{20pt}{de son conditionnement~?}
 \crlf\crlf
 \startalignment[middle]
\monoI{12pt}{La }\monoBI{20pt}{volonté}\crlf \monoI{12pt}{est résistance }%\crlf
\monoI{12pt}{et l'}\monoBI{16pt}{action}\crlf
\monoI{12pt}{ qui en découle }%\crlf
\monoI{12pt}{engendre }\crlf
\monoBI{16pt}{désordre} \monoI{12pt}{et }\monoBI{16pt}{souffrance},\crlf
\monoI{12pt}{ tant intérieurs qu'extérieurs}
 \stopalignment
\crlf\crlf
\serif{20pt}{Tout} \serif{36pt}{conflit}\crlf
\serif{20pt}{est une perte d'}\serifB{26pt}{énergie}
\crlf\crlf
 \startalignment[left]
\mono{12pt}{Toutes nos} \serifB{20pt}{actions}\crlf \mono{12pt}{sont basées sur la} \serifB{20pt}{pensée}\crlf
 \mono{12pt}{et sont donc}\crlf
  \serifB{14pt}{limitées},\crlf
\serifB{16pt}{fragmentaires},\crlf\serifB{20pt}{incomplètes}
   \stopalignment
 \stoptext
